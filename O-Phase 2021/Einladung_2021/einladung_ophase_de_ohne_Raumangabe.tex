\documentclass[nkz,einrichtung,usemycontact]{tucletter2019}


\ifxetex
\usepackage{polyglossia}
\setmainlanguage[spelling=new,babelshorthands=true]{german}
\else
\ifluatex
\usepackage{polyglossia}
\setmainlanguage[spelling=new,babelshorthands=true]{german}
\else
\usepackage[utf8]{inputenc}
\usepackage[T1]{fontenc}
\usepackage[ngerman]{babel}
\fi\fi


\setkomavar{subject}{Herzlich Willkommen - Einladung zur Orientierungsphase 2021}
%\setkomavar{myref}{Akztenzeichen}
%\setkomavar{myfunction}{Musterposition}	% nur wenn notwendig
\setkomavar{date}{31.07.2021}		% am besten immer festes Datum eintragen


\begin{document}	
\begin{letter}{%
\, \\
}

\opening{Liebe Erstis,}

wir möchten Euch hiermit im Namen aller Studierenden der Fakultät für Mathematik recht herzlich begrüßen.
Um Euch den Start ins Studium so angenehm wie möglich zu gestalten, haben wir -- die Vertreter aller Mathematikstudierenden in Chemnitz -- ein kleines Programm für Euch zusammengestellt. Hiermit laden wir Euch herzlich ein zur \\[10pt]

\hspace*{\fill} \Large Orientierungsphase \hspace*{\fill} \\
\hspace*{\fill} Montag, 04.\,10.\,2021, bis Freitag, 08.\,10.\,2021. \normalsize \hspace*{\fill} \\[12pt]

In der Orientierungsphase erwarten Euch normalerweise ein Grill- und ein Spieleabend, eine Campus-, eine Stadt- und eine Kneipentour sowie Einstiegskurse in die wissenschaftliche Mathematik, die Euch auf Eure ersten richtigen Vorlesungen vorbereiten sollen. Aufgrund der aktuellen Situation mit der Corona-Pandemie ist jedoch bei vielen Aktivitäten noch nicht klar, wie sie in dem entsprechenden Zeitraum stattfinden dürfen, können und werden. Wir werden versuchen so viele Veranstaltungen wie möglich in Präsenz unter den zu diesem Zeitpunkt geltenden Beschränkungen durchzuführen und hoffen, dass wir uns in der Orientierungsphase persönlich kennenlernen können. \\[10pt] 

\bfseries Deshalb behaltet bitte unbedingt im Vorfeld unsere Website \\[6pt]
\hspace*{\fill} \Large \itshape \textbf{www.tu-chemnitz.de/fsrmathe} \upshape \normalsize \hspace*{\fill} \\[6pt]
im Blick, auf der Ihr aktuelle Informationen und den vorläufigen Zeitplan findet. Dort werden wir angeben, wie welche Veranstaltungen unter welchen Beschränkungen an welchen Orten stattfinden. \mdseries \\[12pt]

Falls Ihr Fragen habt, schreibt uns gerne eine E-Mail an \itshape\textbf{fsrmathe@tu-chemnitz.de}\upshape!

Bis dahin wünschen wir Euch noch einen schönen Sommer und freuen uns auf Euer Erscheinen.

Euer Fachschaftsrat Mathematik

\end{letter}

\end{document}
