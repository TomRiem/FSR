%Dieses Dokument soll zum Rumspielen dienen, falls Sie noch keine Erfahrung mit Latex haben. Am besten den Quelltext und die Ausgabe zusammen angucken.
%Alles hinter dem ``%'' Zeichen ist Kommentar und wird beim Kompilieren nicht berücksichtigt.

%Den folgenden Kram erstmal überspringen, von hier...
\documentclass{article}

\usepackage{amsmath,amssymb,amsthm,graphicx}
\usepackage{hyperref}
\usepackage{cleveref}
\usepackage{german, tikz}
\usepackage{enumitem}
\usepackage[utf8]{inputenc}

\newtheorem{lemma}{Lemma}
\newtheorem{theorem}[lemma]{Theorem}
\newtheorem{satz}[lemma]{Satz}
\newtheorem{koro}[lemma]{Korollar}
\newtheorem{bem}[lemma]{Bemerkung}
\newtheorem{definition}[lemma]{Definition}
\newenvironment{beweis}{\begin{proof}[Beweis.]}{\end{proof}}
\numberwithin{lemma}{section}

\setlength{\parindent}{0pt}

\newcommand{\pa}{\par \vspace{2mm} }
\newcommand{\foral}{ \ \forall }
\newcommand{\summe}[2]{ \sum_{#1}^{#2} }

\renewcommand{\P}{\mathcal{P}}
\newcommand{\C}{\mathcal{C}}
\newcommand{\N}{\mathbb{N}}
\renewcommand{\H}{\mathcal{H}}
\newcommand{\D}{\mathcal{D}}
\newcommand{\F}{\mathcal{F}}
%...bis hier.



%Hier kann man Titel, Namen etc. eintragen.
\title{Elementare Funktionen}

\author{Michael Schmischke\\
Technische Universit\"at Chemnitz
}

\begin{document}
\maketitle %Titelzeile erzeugen

%Das Abstract ist eine kurze Zusammenfassung des Inhalts.
\begin{abstract}
	\noindent Einf\"uhrung in Grundlagen der Analysis. 
\end{abstract}

%Jetzt kommen ein paar gebräuchliche Umgebungen. Sie werden automatisch numeriert.

\section{Abbildungen}

Wir betrachten Funktionen (bzw. Abbildungen) um Mengen miteinander in Beziehung zu setzen.

\begin{definition}\label{def}
Seien $D$ und $W$ nichtleere Mengen. Eine \textbf{Abbildung} \[ f: D \to W, x \mapsto f(x) \] ist eine eindeutige Zuordnung der \textbf{Definitionsmenge} $D$ zum \textbf{Wertebereich} $W$. Dies ist gleichbedeutend damit, dass jedem Element $x \in D$ ein Element $y := f(x) \in W$ zugeordnet wird.
\end{definition}

\begin{bem}
Eine Abbildung ist i.A. keine eineindeutige Zuordnung, d.h. es kann ein $y \in W$ geben mit $\forall x \in D : f(x) \neq y$.
\end{bem}

\begin{definition}
Zunächst möchten wir einige wichtige Begriffe im Zusammenhang mit Abbildungen definieren. Sei dafür $f$ wie in \cref{def} gegeben.

\begin{tabular}{lll}
 für $x \in D$ & heißt $f(x)$ & Bild von $x$ \\
 für $A \subset D$ & heißt $f(A)$ & Bild(menge) von $A$ \\
 für $A = D$ & heißt $f(A)=f(D)$ & Bild von $f$ \\
 für $y \in W$ & heißt $f^{-1}(y)$ & Urbild von $y$, Faser über $y$ \\
 für $B \subset W$ & heißt $f^{-1}(B)$ & Urbild von $B$ 
\end{tabular}

Mengenklammern werden bei Urbildern häufig weggelassen. Das Urbild stellt allerdings keine Funktion von W nach D dar. Man könnte sie allerdings als Funktion von der Potenzmenge von W in die Potenzmenge von D auffassen.
\end{definition}

\begin{definition}
Eine Teilmenge $\Gamma \subset A \times B$ heißt \textbf{Graph} einer Abbildung, wenn
\begin{enumerate}[label={\alph*)}]
\item $\foral x \in A \, \exists y \in B : (x,y) \in \Gamma$ und
\item $(x,y),(x,y')\in\Gamma \Rightarrow y = y'$
\end{enumerate}
erfüllt sind.
\end{definition}

\section{Eigenschaften}

In diesem Abschnitt betrachten wir grundlegende Eigenschaften von Abbildungen und setzen sie in Beziehung zueinander.

\begin{definition}
Eine Abbildung $f: D \to W, x \mapsto f(x)$ heißt \textbf{injektiv} (eineindeutig), wenn eine der folgenden äquivalenten Bedingungen gilt.
\begin{enumerate}
\item $\forall y \in W$ existiert höchstens ein $x \in D$ mit $f(x)=y$
\item $|f^{-1}(\{y\})| \leq 1 \foral y \in W$
\item $x \neq x' \Rightarrow f(x) \neq f(x')$
\item $f(x) = f(x') \Rightarrow x = x'$
\end{enumerate}
Eine Abbildung $f: D \to W, x \mapsto f(x)$ heißt \textbf{surjektiv} (Abbildung auf $W$), wenn eine der folgenden äquivalenten Bedingungen gilt.
\begin{enumerate}
\item $\forall y \in W \, \exists x \in D : f(x)=y$
\item $f^{-1}(\{y\}) \neq \emptyset \foral y \in W$
\item $f(D) = W$
\end{enumerate}
Eine Abbildung $f: D \to W, x \mapsto f(x)$ heißt \textbf{bijektiv} (eineindeutige Abbildung auf $W$), wenn eine der folgenden äquivalenten Bedingungen gilt.
\begin{enumerate}
\item $\forall y \in W \, \exists ! x \in D : f(x)=y$
\item $|f^{-1}(\{y\})| = 1 \foral y \in W$
\item f ist injektiv und surjektiv
\end{enumerate}
\end{definition}

\begin{satz}
Seien $D$ und $S$ endliche Mengen mit $|D|=|S|=n\in\mathbb{N}$. Dann gilt für jede Abbildung $f: D \to S$  
surjektiv $\Leftrightarrow$ injektiv $\Leftrightarrow$ bijektiv.
\end{satz}

\begin{beweis}
Diesen Satz beweisen wir über einen Ringschluss. Offensichtlich gilt bijektiv $\Rightarrow$ injektiv. \pa
Injektiv bedeutet, jedes Element aus $S$ hat ein Urbild mit $0$ Elementen oder $1$ Element. Ein Urbild kann in unserem Fall aber nicht aus $0$ Elementen bestehen, da es sonst Elemente aus $D$ geben würde, die man nicht abbildet. Damit ist die Abbildung auch surjektiv, wenn sie injektiv ist. \pa
Surjektiv bedeutet, dass jedes Urbild mindestens ein Element enthält. Würde ein Urbild allerdings mehr als $1$ Element enthalten, dann gäbe es ein Element aus $S$ mit einem leeren Urbild, was ein Widerspruch ist. Damit ist die Abbildung bijektiv, wenn sie surjektiv ist.
\end{beweis}

\section{Komposition und Umkehrabbildung}

\begin{definition}
Seien $f: A \to B$ und $g: B \to C$ zwei Abbildungen. Die \textbf{Komposition} oder das Produkt $g \circ f$ ist die Abbildung, welche durch Hintereinanderausführung von f und g entsteht:
\begin{align*}
	g \circ f: A \to C \\
	x \mapsto g(f(x))\\
	(g \circ f)(x) = g(f(x))
\end{align*} \par
\end{definition}

Beispiel: $g(x) = x^2, f(x) = e^x, (g \circ f)(x) = g(f(x)) = e^{2x}$

\begin{definition}
Eine Abbildung $f$ heißt \textbf{Identität} auf einer Menge $D$, auch $id_D$ genannt, wenn sie jedes Element von D auf sich selbst abbildet. 
\end{definition}

\begin{satz}
Sei $D$ eine Menge. Die Identität $id_D: D \to D, x \mapsto x$ ist eine bijektive Abbildung.
\end{satz}

\begin{beweis}
Wir verwenden die äquivalente Bedingung 2 aus Kapitel 2. Sei $d\in D$ beliebig. Da jedes Element auf sich selbst abgebildet wird, gilt $id_D^{-1}(\{d\}) = \{d\}$ und somit ist die Bedingung erfüllt.
\end{beweis}

\begin{satz}
Sei $f: D \to W, x \mapsto f(x)$ eine bijektive Abbildung. Dann gibt es genau eine Abbildung $g: W \to D$ mit der Eigenschaft
\begin{equation*}
f \circ g = id_W,\, g \circ f = id_D
\end{equation*}
Diese eindeutige Abbildung wird auch Umkehrabbildung oder inverse Abbildung genannt und oft mit $f^{-1}$ bezeichnet. Sie ist aber nicht mit dem Urbild zu verwechseln.
\end{satz}

\begin{beweis}
%$f$ ist eine bijektive Abbildung $\Rightarrow \forall b \in B \, \exists ! a \in A : f^{-1}(\{b\})=\{a\}$ \\
Wir definieren $h: W \to D, y \mapsto f^{-1}(\{y\})$, wobei mit $f^{-1}(\{y\})$ das eindeutige Element der Menge gemeint ist. Nun gilt
\begin{equation*}
f(h(y)) = f(f^{-1}(\{y\})) = y.
\end{equation*}
$h$ ist also unsere gesuchte Abbildung $g$, da offenbar gilt $f \circ h = id_W$. Analog zeigt man $id_D$. \\
Angenommen $g$ und $\tilde{g}$ haben beide die Eigenschaft. Dann ist für beliebiges $x \in D$ mit $y := f(x)$
\begin{equation*}
g(f(x)) = g(y) = \tilde{g}(y) = \tilde{g}(f(x)).
\end{equation*}
\end{beweis}

\begin{satz}
Die Umkehrabbildung einer bijektiven Abbildung ist bijektiv.
\end{satz}

\end{document}