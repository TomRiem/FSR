\documentclass[nkz,einrichtung,usemycontact]{tucletter2019}


\ifxetex
\usepackage{polyglossia}
\setmainlanguage[spelling=new,babelshorthands=true]{english}
\else
\ifluatex
\usepackage{polyglossia}
\setmainlanguage[spelling=new,babelshorthands=true]{english}
\else
\usepackage[utf8]{inputenc}
\usepackage[T1]{fontenc}
\usepackage[ngerman]{babel}
\fi\fi


\setkomavar{subject}{Welcome - Invitation to the Orientation Period 2019}
\setkomavar{date}{30.06.2019}		% am besten immer festes Datum eintragen


\begin{document}	
\begin{letter}{%
\, \\
}

\opening{Dear Freshman,}

We would like to welcome you on behalf of all students of the faculty of mathematics. To provide you a pleasant start in your studies we – the representatives of all students of mathematics in Chemnitz – have prepared a little program for you. We would like to invite you to the\\[12pt]

\hspace*{\fill} \Large Orientation Period \hspace*{\fill} \\
\hspace*{\fill} Monday, 10/07/2019 to Friday, 10/11/2019. \normalsize \hspace*{\fill} \\[16pt]

\bfseries Most important is the event on Monday, 7 th October at 9 a.m. in room
2/N 010. This room is located in the ground floor of the central lecture building
at the Reichenhainer Straße. At this place the faculty and the Fachschaftsrat
will introduce themselves. Furthermore, we will provide the first important
information.\mdseries

\medskip

During the orientation period you may look forward to a tour through the city and the
campus, introduction courses for mathematicians, to prepare you for your first real lectures and the celebratory matriculation on Thursday, 11th October. You also don’t need to stay at home in the evenings, since there will be a BBQ and tours to some bars and student clubs.

\medskip

Detailed information and the time table will be available on\\[8pt]
\hspace*{\fill} \itshape\textbf{www.tu-chemnitz.de/fsrmathe} \upshape \hspace*{\fill} \\[8pt]
or at the meeting on 10/07/2019 at 9 a.m.

\medskip

Until then we wish you a pleasant summer and are looking forward to seeing you.

\medskip

Yours,

Fachschaftsrat Mathematik

\end{letter}

\end{document}