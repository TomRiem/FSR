\documentclass[nkz,einrichtung,usemycontact]{tucletter2019}


\ifxetex
\usepackage{polyglossia}
\setmainlanguage[spelling=new,babelshorthands=true]{english}
\else
\ifluatex
\usepackage{polyglossia}
\setmainlanguage[spelling=new,babelshorthands=true]{english}
\else
\usepackage[utf8]{inputenc}
\usepackage[T1]{fontenc}
\usepackage[ngerman]{babel}
\fi\fi


\setkomavar{subject}{Welcome - Invitation to the Orientation Period 2020}
\setkomavar{date}{31.07.2020}		% am besten immer festes Datum eintragen


\begin{document}	
\begin{letter}{%
\, \\
}

\opening{Dear Freshman,}

We would like to welcome you on behalf of all students of the faculty of mathematics. To provide you a pleasant start in your studies we – the representatives of all students of mathematics in Chemnitz – have prepared a little program for you. We would like to invite you to the\\[12pt]

\hspace*{\fill} \Large Orientation Period \hspace*{\fill} \\
\hspace*{\fill} Monday, 10/05/2020 to Friday, 10/09/2020. \normalsize \hspace*{\fill} \\[16pt]

\bfseries Most important is the event on Monday, 5 th October at 10:30 a.m. in room
2/N 112. This room is located on the first floor of the central lecture building
at the Reichenhainer Straße. At this place the faculty and the "Fachschaftsrat"
will introduce themselves. Furthermore, we will provide the first important
information.\mdseries

\medskip

During the orientation period you may also look forward to a tour through the city, the
campus, student clubs and some bars, a BBQ and introduction courses for mathematicians, to prepare you for your first real lectures. Due to Corona it is not known yet, which of these activities can and will take place and how.

\medskip

Therefore, please keep an eye on our website\\[8pt]
\hspace*{\fill} \itshape\textbf{www.tu-chemnitz.de/fsrmathe,} \upshape \hspace*{\fill} \\[8pt]
where we will share further information and the the preliminary schedule for the week.

\medskip

Until then we wish you a pleasant summer and are looking forward to seeing you.

\medskip

Yours,

Fachschaftsrat Mathematik

\end{letter}

\end{document}