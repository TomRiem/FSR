\documentclass[nkz,einrichtung,usemycontact]{tucletter2019}


\ifxetex
\usepackage{polyglossia}
\setmainlanguage[spelling=new,babelshorthands=true]{german}
\else
\ifluatex
\usepackage{polyglossia}
\setmainlanguage[spelling=new,babelshorthands=true]{german}
\else
\usepackage[utf8]{inputenc}
\usepackage[T1]{fontenc}
\usepackage[ngerman]{babel}
\fi\fi


\setkomavar{subject}{Stellungnahme zu dem Lehrpreis für den Lernförderlichen Einsatz digitaler Technologien}
%\setkomavar{myref}{Akztenzeichen}
%\setkomavar{myfunction}{Musterposition}	% nur wenn notwendig
\setkomavar{date}{20.07.2021}		% am besten immer festes Datum eintragen


\begin{document}	
\begin{letter}{%
\, \\
}

\opening{Sehr geehrter Herr Prof. Dr. Pietschmann,}

nachfolgend die Stellungnahme des Fachschaftsrates Mathematik bzgl. Ihrer Nominierung für den oben genannten Lehrpreis.

„Die Umsetzung der digitalen Lehre vom Prof. Pietschmann wurde mehrfach von den Studierenden gelobt. Das zeigen auch die sehr positiven Bewertungen in der Lehrevaluation. Durch die von Prof. Pietschmann organisierten technischen Voraussetzungen, war es ihm möglich die Vorlesung „an der Tafel“, was typisch für die Mathematik Vorlesungen ist, zu halten. Die Vorlesungen und Übungen fanden zu den im Vorlesungsverzeichnis festgelegten Zeiten im BigBlueButton-Raum statt. Somit gelang es Prof. Pietschmann die digitalen und analogen Ansätzen bestens zu kombinieren. Die Studierenden hatten immer die Möglichkeit eigene Kamera anzumachen und aktiv mitzuwirken, was von vielen Teilnehmenden genutzt wurde. Damit konnte eine sehr gute Kommunikationsgrundlage zwischen dem Lehrenden und den Studierenden geschaffen werden, was für eine sehr angenehme Kursatmosphäre im Laufe des ganzen Semesters sorgte. Die Unklarheiten konnten schnell und einfach in der Veranstaltung beseitigt werden und Verständnisfragen geklärt. Die Übung war stets mit der Vorlesung abgestimmt. Die Übungsblätter waren frühzeitig verfügbar, so dass die Studierenden sich selbständig mit dem Thema auseinander setzen konnten. Die Ergebnisse wurde dann in den Übungseinheiten von den Studierenden vorgestellt und gemeinsam ausführlich besprochen.“

Mit freundlichen Grüßen

Tom-Christian Riemer 
Sitzungsleiter des Fachschaftsrates Mathematik

\end{letter}

\end{document}
